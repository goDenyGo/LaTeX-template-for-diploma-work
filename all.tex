%= = = = = = = = = = = = = = = = = = = = = = = = = = = = = = = = = = = = = = = = = = =
%       Шаблон розроблений Герасимчуком Назаром (nazar.gerasymchuk@gmail.com)
%   доступний за адресою: https://github.com/troyane/LaTeX-template-for-diploma-work
%- - - - - - - - - - - - - - - - - - - - - - - - - - - - - - - - - - - - - - - - - - - 
%  Головний файл all.tex
%= = = = = = = = = = = = = = = = = = = = = = = = = = = = = = = = = = = = = = = = = = =

\documentclass[a4paper,14pt]{extreport} % Розмір паперу А4, шрифт 14 пунктів
\usepackage[T2A]{fontenc}
\usepackage[english,ukrainian]{babel}
\usepackage{ucs}
\usepackage[utf8]{inputenc} % включємо кодування utf8 в *NIX (cp1251 в Windows)
\usepackage{amssymb,amsfonts,amsmath,mathtext,cite,enumerate,float} % підключаємо пакети розширень
\usepackage{listings} % для вихідних кодів
\usepackage{alltt}  
\usepackage{algorithmic}
\usepackage{indentfirst} % для абзаців
\usepackage[pdftex]{graphicx}

\renewcommand{\rmdefault}{cmr}
\renewcommand{\sfdefault}{ftx}
\renewcommand{\ttdefault}{cmtt}

\makeatletter

%\bibliographystyle{unsrt}
\bibliographystyle{gost780s} % встановлюємо тип бібліографії
%\def\BibUrl#1{#1}

\renewcommand{\@biblabel}[1]{#1.} % змінюємо формат нумерації бібліографії на "цифра."

\makeatother

\usepackage{geometry} % Меняєм поля сторінки
\geometry{left=3cm}
\geometry{right=1cm}
\geometry{top=2cm}
\geometry{bottom=2cm}

\usepackage{setspace} % інтерліньяж
\onehalfspacing

%ОТСАННІ ПРАВКИ
\makeatletter

\renewcommand*\@makechapterhead[1]{
  {\parindent \z@ \center \normalfont \bfseries
    \LARGE \textsc
    \@chapapp{} \thechapter. #1
    \nopagebreak
    \vspace{40pt}
  }}
\makeatother

% Змінюємо скрізь перелічення на наступні "цифра.цифра.":
\renewcommand{\theenumi}{\arabic{enumi}.} 	
\renewcommand{\labelenumi}{\arabic{enumi}.} 
\renewcommand{\theenumii}{.\arabic{enumii}.} 
\renewcommand{\labelenumii}{\arabic{enumi}.\arabic{enumii}.} 
\renewcommand{\theenumiii}{.\arabic{enumiii}.} 
\renewcommand{\labelenumiii}{\arabic{enumi}.\arabic{enumii}.\arabic{enumiii}.}

\righthyphenmin=2 % мінімальна к-ть символів для переносу
\sloppy

\begin{document}

	%= = = = = = = = = = = = = = = = = = = = = = = = = = = = = = = = = = = = = = = = = = =
%       Шаблон розроблений Герасимчуком Назаром (nazar.gerasymchuk@gmail.com)
%   доступний за адресою: https://github.com/troyane/LaTeX-template-for-diploma-work
%- - - - - - - - - - - - - - - - - - - - - - - - - - - - - - - - - - - - - - - - - - - 
%  Файл титульної сторінки title_page.tex
%= = = = = = = = = = = = = = = = = = = = = = = = = = = = = = = = = = = = = = = = = = =

\newpage
\begin{titlepage}
\begin{center}
МІНІСТЕРСТВО ОСВІТИ І НАУКИ, МОЛОДІ ТА СПОРТУ УКРАЇНИ \\
\vspace{1em}
ВОЛИНСЬКИЙ НАЦІОНАЛЬНИЙ УНІВЕРСИТЕТ \\ ІМЕНІ ЛЕСІ УКРАЇНКИ \\
\vspace{2em}
\textbf{Кафедра \LaTeX -логії}
\end{center}
\vspace{5em}

\begin{center}
\Large{\textbf{\textsc{Курсова робота}}}\\  
\LARGE{\textsc{\textbf{Дослідження життя жаби звичайної в атеїстичному світі}}}
\end{center}

\vspace{2em}

\begin{flushleft}
\hspace{9cm}Виконав студент ХХ групи\\
\hspace{9cm}\LaTeX -нічного ф-ту\\
\hspace{9cm}спеціальності <<\LaTeX -ніка>>\\
\hspace{9cm}Іванов Іван Іванович\\

\vspace{2em}

\hspace{9cm}Науковий керівник\\
\hspace{9cm}канд.~фіз-мат.~наук,\\
\hspace{9cm}Петренко Петро Петрович

\end{flushleft}

\vspace{\fill}

\begin{center}
Луцьк -- 2011
\end{center}				
\end{titlepage} 		% Титулка
	\newpage
\tableofcontents





 	% Зміст
	%= = = = = = = = = = = = = = = = = = = = = = = = = = = = = = = = = = = = = = = = = = =
%       Шаблон розроблений Герасимчуком Назаром (nazar.gerasymchuk@gmail.com)
%   доступний за адресою: https://github.com/troyane/LaTeX-template-for-diploma-work
%- - - - - - - - - - - - - - - - - - - - - - - - - - - - - - - - - - - - - - - - - - - 
%  Файл глосарію gloss.tex
%= = = = = = = = = = = = = = = = = = = = = = = = = = = = = = = = = = = = = = = = = = =

\newpage
\chapter*{Глосарій}
\addcontentsline{toc}{chapter}{\textsc{Глосарій}}

\textbf{\textsc{\LaTeX}} (вимовляється «латех») -- мова розмітки даних та пакет макросів \TeX для високоякісного оформлення документів, створений Леслі Лампортом (англ. Leslie Lamport). Вважається стандартом де-факто для підготовки математичних і технічних текстів для публікації в наукових виданнях. \\

\textbf{\textsc{Безхвості або жаби (Anura, Salientia, Ecaudata)}} -- ряд земноводних. Більшість -- корисні тварини, деякі є об`єктом промислу. \\

\textbf{\textsc{Атеїзм або войовниче безбожжя (від грец. $\alpha\theta\varepsilon\sigma\varsigma$, безбожний)}} -- світогляд, вчення про життя, філософське вчення, однією з основ якого є заперечення існування будь-яких богів духів, інші <<нематеріальні>> істоти тощо. \\









			% Глосарій
	
	%= = = = = = = = = = = = = = = = = = = = = = = = = = = = = = = = = = = = = = = = = = =
%       Шаблон розроблений Герасимчуком Назаром (nazar.gerasymchuk@gmail.com)
%   доступний за адресою: https://github.com/troyane/LaTeX-template-for-diploma-work
%- - - - - - - - - - - - - - - - - - - - - - - - - - - - - - - - - - - - - - - - - - - 
%  Файл вступу gloss.tex
%= = = = = = = = = = = = = = = = = = = = = = = = = = = = = = = = = = = = = = = = = = =


\newpage
\chapter*{\textsc{Вступ}}
\addcontentsline{toc}{chapter}{\textsc{Вступ}}

Життя жаби звичайної -- нелагка штука, з огляду на атеїстичні погляди нашого суспільства. Як писав один із дослідників жаби, Піонер Гриша: <<Жаба живе в саду, живиться продуктами народного господарства, тому жаба – шкідник, її треба знищувати. Наш народ вже даво оголосив війну жабі>>.
 
\textbf{Мета дослідження} -- дослідити життя жаби глибше.

Поставлена мета передбачає виконання таких \textbf{завдань:}

\begin{enumerate}
	\item Вивчення літератури про жаб та атеїзм.
	\item Дослідити ставлення Дарвіна до жаб.
	\item Отримання та аналіз результатів досліджень.
\end{enumerate} 			% Вступ

	%= = = = = = = = = = = = = = = = = = = = = = = = = = = = = = = = = = = = = = = = = = =
%       Шаблон розроблений Герасимчуком Назаром (nazar.gerasymchuk@gmail.com)
%   доступний за адресою: https://github.com/troyane/LaTeX-template-for-diploma-work
%- - - - - - - - - - - - - - - - - - - - - - - - - - - - - - - - - - - - - - - - - - - 
%  Файл першої частини chapter1.tex
%= = = = = = = = = = = = = = = = = = = = = = = = = = = = = = = = = = = = = = = = = = =

\newpage
\chapter{\textsc{Основні\- поняття\- з життя жаби}}

\section{Походження}
Найдавнішим викопним безхвостими земноводними є протобатрахус (Protobatrachus massinoti) з тріасу Мадагаскару (який ще мав хвіст). Зоологи з Манчестерського університету виявили в Коста-Ріка, у заповіднику вологих тропічних лісів Монтеверде, самицю маленької деревної жаби виду Іsthmohyla rіvularіs, що вважався вимерлим ще 20 років тому. Вчені зазначають, що іще в 2007 році в Коста-Ріка була помічена чоловіча особина жаби цього виду. А виявлення тепер жіночої особини дозволяє вченим припустити, що ці земноводні розмножуються й здатні вижити.

Спочатку дослідникам вдалося знайти самця жаби, самостійно наслідуючи звук жаби, а потім голова заказника Тропічного наукового центру Монтеверде Луїс Обандо виявив і маленьку самицю жаби, що сиділа на листі. Герпетолог музею манчестерского університету Ендрю Грій заявив, що виявлення цієї жаби є вершиною всієї його кар'єри.

<<Зараз, коли ми знаємо, що обидві статі цього виду жаби існують у дикій природі, важливо дослідити особливості існування цього виду й докласти максимум зусиль для його збереження>>, - сказав Грій.

Він також повідомив, що виявлена самиця жаби, довжина тіла якої 2,5 сантиметри, була коричневою з маленькими зеленим плямами. Знайти її було вкрай важко, тому що чоловічі особини часто видають голосні закличні звуки, у той час як самки роблять це досить рідко.

\section{Поширення}
Поширені безхвості земноводні в усіх ландшафтно-географічних зонах, крім полярних областей (трав'яна жаба заходить і за Полярне коло) та безводних пустель. Найбільше безхвостих земноводних у Тропічній Америці. В Україні трапляються 12 видів.


\section{Класифікація}
Ряд налічує 256 родів та близько 3 500 видів, 16—31 родин: піпові — Pipidae (підродина пазуристі жаби, піпи), круглоязикові, часничниці, дереволази, квакші, рінодерми, свистуни, ропухові, вузькороті жаби,Скляні жаби та інші.

  		% Розділ 1
	%= = = = = = = = = = = = = = = = = = = = = = = = = = = = = = = = = = = = = = = = = = =
%       Шаблон розроблений Герасимчуком Назаром (nazar.gerasymchuk@gmail.com)
%   доступний за адресою: https://github.com/troyane/LaTeX-template-for-diploma-work
%- - - - - - - - - - - - - - - - - - - - - - - - - - - - - - - - - - - - - - - - - - - 
%  Файл другої частини chapter1.tex
%= = = = = = = = = = = = = = = = = = = = = = = = = = = = = = = = = = = = = = = = = = =

\newpage

\chapter{\textsc{Проектування\- і розробка\- штучного середовища для проживання жаби звичайної}}

\section{Постановка\- проблеми\- сворення ідеального середовища}

\section{Впилив атеїстичних поглядів на життя жаби}

 		% Розділ 2
	%= = = = = = = = = = = = = = = = = = = = = = = = = = = = = = = = = = = = = = = = = = =
%       Шаблон розроблений Герасимчуком Назаром (nazar.gerasymchuk@gmail.com)
%   доступний за адресою: https://github.com/troyane/LaTeX-template-for-diploma-work
%- - - - - - - - - - - - - - - - - - - - - - - - - - - - - - - - - - - - - - - - - - - 
%  Файл висновків results.tex
%= = = = = = = = = = = = = = = = = = = = = = = = = = = = = = = = = = = = = = = = = = =

\newpage
\chapter*{\textsc{Висновки}}
\addcontentsline{toc}{chapter}{\textsc{Висновки}}

В ході виконання курсової роботи було опрацьовано велику кількість літератури про жаб і про атеїзм.
Головне, що написання курсової роботи приносило радість людям і жабам. 		% Висновки
	%= = = = = = = = = = = = = = = = = = = = = = = = = = = = = = = = = = = = = = = = = = =
%       Шаблон розроблений Герасимчуком Назаром (nazar.gerasymchuk@gmail.com)
%   доступний за адресою: https://github.com/troyane/LaTeX-template-for-diploma-work
%- - - - - - - - - - - - - - - - - - - - - - - - - - - - - - - - - - - - - - - - - - - 
% Файл бібліографії literature.tex в якому засобами BibTeX з файлу my_bibliography.bib
%  підвантажується перелік л-ри, на яку є вказівники в тексті.
%= = = = = = = = = = = = = = = = = = = = = = = = = = = = = = = = = = = = = = = = = = =

\newpage
\addcontentsline{toc}{chapter}{\textsc{Бібліографія}}
\bibliography{my_bibliography}

 		% Використана лытература
	%= = = = = = = = = = = = = = = = = = = = = = = = = = = = = = = = = = = = = = = = = = =
%       Шаблон розроблений Герасимчуком Назаром (nazar.gerasymchuk@gmail.com)
%   доступний за адресою: https://github.com/troyane/LaTeX-template-for-diploma-work
%- - - - - - - - - - - - - - - - - - - - - - - - - - - - - - - - - - - - - - - - - - - 
%  Файл додатків appendix.tex
%= = = = = = = = = = = = = = = = = = = = = = = = = = = = = = = = = = = = = = = = = = =


\chapter*{\textsc{Додатки}}
\addcontentsline{toc}{chapter}{\textsc{Додатки}}

\subsection*{Частини програмного коду жаби-кіборга}
Вихідний код модуля myTypes.h, в якому описано основну структуру -- точку в тривимірному просторі glPoint3d.
\begin{verbatim}
#include <math.h>
#include <QtOpenGL>
// опис "нуля" термінами OpenGL
#define GL_0 (GLfloat(0.0))

const float EPS = 1.0E-5;		// похибкa 
const float RTOD = 180/M_PI;	// коеф. радіани->градуси

using namespace std;

//ф-ія квадрату
inline GLfloat sqr(GLfloat x){ 
   return (GLfloat) (x*x); 
}  

// ф-ія перевірки x на близкість до 0.0
inline bool isZero(GLfloat x){ 
   return fabsf((float)x)<EPS ? true : false; 
}

class glPoint3d{
// клас точки в 3d з текстурними координатами
public:
   GLfloat x, y, z;
   GLfloat tx, ty;

   glPoint3d(GLfloat nx, GLfloat ny, GLfloat nz, GLfloat ntx, 
   				GLfloat nty){
    x=nx; y=ny; z=nz; tx=ntx; ty=nty;
};  
// занулення
void makeZeros(){x=GL_0; y=GL_0; z=GL_0; tx=GL_0; ty=GL_0;}
// нормування
void normalize(){
   GLfloat l = 1.0f/sqrt(sqr(x) + sqr(y) + sqr(z));
   if(!isZero(l)){
    x *= l;
    y *= l;
    z *= l;
   }else{ x=GL_0; y=GL_0; z=GL_0; }
};
// пошук відстані до other
GLfloat to(glPoint3d *other){ 
   return (GLfloat) sqrt(sqr(x-other->x) + 
            sqr(y-other->y)+sqr(z-other->z)); 
  };
};
\end{verbatim} 		% Додатки
	
\end{document}