%= = = = = = = = = = = = = = = = = = = = = = = = = = = = = = = = = = = = = = = = = = =
%       Шаблон розроблений Герасимчуком Назаром (nazar.gerasymchuk@gmail.com)
%   доступний за адресою: https://github.com/troyane/LaTeX-template-for-diploma-work
%- - - - - - - - - - - - - - - - - - - - - - - - - - - - - - - - - - - - - - - - - - - 
%  Файл глосарію gloss.tex
%= = = = = = = = = = = = = = = = = = = = = = = = = = = = = = = = = = = = = = = = = = =

\newpage
\chapter*{Глосарій}
\addcontentsline{toc}{chapter}{\textsc{Глосарій}}

\textbf{\textsc{\LaTeX}} (вимовляється «латех») -- мова розмітки даних та пакет макросів \TeX для високоякісного оформлення документів, створений Леслі Лампортом (англ. Leslie Lamport). Вважається стандартом де-факто для підготовки математичних і технічних текстів для публікації в наукових виданнях. \\

\textbf{\textsc{Безхвості або жаби (Anura, Salientia, Ecaudata)}} -- ряд земноводних. Більшість -- корисні тварини, деякі є об`єктом промислу. \\

\textbf{\textsc{Атеїзм або войовниче безбожжя (від грец. $\alpha\theta\varepsilon\sigma\varsigma$, безбожний)}} -- світогляд, вчення про життя, філософське вчення, однією з основ якого є заперечення існування будь-яких богів духів, інші <<нематеріальні>> істоти тощо. \\









